\documentclass{article}
\usepackage[utf8]{inputenc}
\usepackage{hyperref}

\newcommand{\shellcmd}[1]{\\\indent\indent\texttt{\footnotesize\# #1}\\}

\title{Installation Manual - mCloud\\Version 1}
\author{Koen van der Kruk}
\date{\today}

\begin{document}

\maketitle

\tableofcontents

\newpage
\section{Introduction}
This document will describe the installation of a custom created cloud storage by Koen van der Kruk in collaboration with Nanquanu. 

This tutorial can be used for server- and client side usage with mCloud. Choose the right section for your own needs. 

The latest mCloud version can be found at the following place:

\url{https://github.com/koen-github/mCloud}

I am not responsible for any damage that possibly can happen when using this tutorial, including the usage of mCloud. 
I tested all the steps in a virtualbox machine running 'Debian GNU/Linux 8 (jessie) 64-bit'



\textbf{I recommend to firstly try to successfully setup a full working mCloud in a virtualbox, and later try to install it in your normal workspace.
Note: Make sure to read also the text BEFORE you run the unix-command. Sometimes the command has to be changed to fill in the specific needs of your own setup.}



\section{Requirements}
I assume you have at least a Debian distribution running like: Ubuntu, Linux Mint, Xubuntu or Debian. 

\begin{itemize}
    \item A PC running a debian distribution
    \item Internet Connection
    \item Some linux knowledge about running commands in terminal and googling your own problems.
    \item Central server system also containing mCloud
    \item As client and server it's required to have full administrator (root) rights onto your box.
\end{itemize}
\section{Installation}
You can choose what kind of mCLoud installation you want. If you're just a normal user who has been ordered to install mCloud on your local machine, you can choose for Client Side installation.  

\subsection{Installation - Server Side}
As admin you probably know a lot about ubuntu and debian, but I will still explain every step.
Start by opening a new terminal.

\begin{enumerate}
     
    \item Clone the mCloud repository in a new and clean directory.
    \shellcmd{git clone git@github.com:koen-github/mCloud.git}
    \item Change directory into mCloud using
    \shellcmd{cd mCloud/}
    \item First make sure you have all the dependencies installed on your PC:
    \shellcmd{sudo apt-get install sshfs lvm2 cryptsetup openssh-server openssh-client nfs-kernel-server nfs-common portmap acl}
    \item Now install the debian package in the mCloud/ folder. Run this command:
    \shellcmd{sudo dpkg -i mcloud-*.deb}
    
    \item Open a new terminal, so do not yet follow the steps onscreen of the dpkg installer.
    
    \item Create a new directory in your home folder where you want to locate the mCloud encrypted image.
    Change IMAGE\_MCLOUD to something of your own!
    \shellcmd{mkdir IMAGE\_MCLOUD}
    
    \item Go back to the installation window of mCloud, choose 'server' and press Enter.
    \item Go inside the newly created folder of IMAGE\_MCLOUD with: \shellcmd{cd IMAGE\_MCLOUD}
    
    \item Get the full installation path of mCloud using this command \shellcmd{pwd} This will return something like: 
    \shellcmd{/home/koen/IMAGE\_MCLOUD}
    
    \item Paste it into the installation window of mCloud and press Enter.
    
    \item Type into the installation window the image size, like: 100M, 200M or something larger and press Enter
    
    \item Type in a name of your mCloud image and press enter
    
    \item Choose YES to setup mCloud with ACL permissions and press Enter. \textbf{After the installation is finished, you have to follow these steps online:}
    \url{https://help.ubuntu.com/community/FilePermissionsACLs#Enabling_ACLs_in_the_Filesystem}
    
    \textbf{After you followed these steps, you have to run the commands specified on the terminal.}
    
    
    \item Add mcloud-users to your sudoers, so users can create their own groups of people with own repositories and press Enter.
    
    \item Now carefully read the log in the terminal to make sure you finished all the steps. 
    
\end{enumerate}


\subsection{Installation - Client Side}
This is a step for step guide to install mCloud on your local machine as normal user.

\begin{enumerate}
    \item Clone the mCloud repository in a new and clean directory.
    \shellcmd{git clone git@github.com:koen-github/mCloud.git}
    \item Change directory into mCloud using
    \shellcmd{cd mCloud/}
    \item First make sure you have all the dependencies installed on your PC:
    \shellcmd{sudo apt-get install sshfs lvm2 cryptsetup openssh-server openssh-client nfs-kernel-server nfs-common portmap acl}
    \item Now install the debian package in the mCloud/ folder. Run this command:
    \shellcmd{sudo dpkg -i mcloud-*.deb}
    
    \item Open a new terminal, so do not yet follow the steps onscreen of the dpkg installer.
    
    \item Create a new directory in your home folder where you want to locate the mCloud encrypted image.
    Change IMAGE\_MCLOUD to something of your own!
    \shellcmd{mkdir IMAGE\_MCLOUD}
    
    \item Go back to the installation window of mCloud, choose 'client' and press Enter.
    \item Go inside the newly created folder of IMAGE\_MCLOUD with: \shellcmd{cd IMAGE\_MCLOUD}
    
    \item Get the full installation path of mCloud using this command \shellcmd{pwd} This will return something like: 
    \shellcmd{/home/koen/IMAGE\_MCLOUD}
    
    \item Paste it into the installation window of mCloud and press Enter.
    
    \item Type into the installation window the image size, like: 100M, 200M or something larger and press Enter
    
    \item Type in a name of your mCloud image and press enter
    
    \item Now carefully read the log in the terminal to make sure you finished all the steps. 
\end{enumerate}

\subsubsection{Make connection}
To setup a connection to a central server, run this command:  \shellcmd{mcloud assignServer} 

Generate a new SSH key pair on the default location, if you already have one you can send that one out to the central server.

You also have to know the IP of the central server. 

Wait for the server admin to apply your rights on the central server.

\subsubsection{Open Connection}
To connect your local system with the remote central server, it's necessary to first assign a server (see above). If the central server has accepted your SSH key and created a new account on the central server, it's time to fire up the following command:
\shellcmd{mcloud connectClientServer}

This will create a connection between the client and the server.

\subsubsection{Close Connection}

When you don't want be connected anymore to the central server, you can close up the connection.
Use the following command:
\shellcmd{mcloud disconnectClientServer}

\subsubsection{Assign User}

As client it's possible to assign a user to a project you want to work on. You can work together with other users sharing a common git repository. 

Run this command: 
\shellcmd{mcloud assignUser}

\begin{enumerate}
\item Fill in the form, create a repository with a good name, so only containing letters, numbers and underscores.
\item Second specify the list of users you want to work with. The users must be ; separated
\item When the script is finished, reconnect with the Server and the Client. After that you can work together in a new repository.
\end{enumerate}







\newpage

\subsection{Delete mCloud}

If you want to remove mCloud from your system, without removing the created image and ssh keys, you can just run the following linux command:
\shellcmd{sudo dpkg -r mcloud}


\subsection{Open Encrypted Container}

After successfully installing the mCloud binaries to your local system, you can run the command:
\shellcmd{mcloud openEncryptedContainer}
This will open the encrypted file container for further usage with the mCloud system. Make sure you have the right settings in your mCloud configuration file. If you installed everything according this tutorial, the settings are already correct.

\subsection{Closing Encrypted Container}
If you want to make sure no one can access the encrypted container it's best to just close it. 

The following command will close the encrypted container.
\shellcmd{mcloud closeEncryptedContainer}




\end{document}
